%{\Large %Remove comment '%' to get a 'Large' font introduction section and abstract
\begin{abstract}
In 2010, the number of mobile devices in the world surpassed the number of personal computers. Mobile devices carry confidential data, both personal and corporate. As a result, mobile devices have become a lucrative target for attackers, and privacy and security of these devices have become a vital issue since. The existing access control mechanisms in most devices, which relies on install time permission granting, is too restrictive and inadequate, since it cannot factor in the context of the device and the user. In other words, the access granted a subject can change based on the context of the device. In this paper we present Mithril, a context-driven dynamic policy-based model for defining access control on mobile devices. We describe the design of the system that captures these policy rules which are defined using Semantic Web technologies, protects a user's mobile device by executing these rules. Specifically, this paper address the question of how such policies can be obtained. We describe an iterative process that helps users in reaching their access control policies by informing them about potential policy violations on their devices. Our evaluation shows that such an iterative process is capable of capturing a specific user's policy.
\end{abstract}
% A category with the (minimum) three required fields
\category{D.4.6}{Operating Systems}{Security and Protection}[Access Controls]
%A category including the fourth, optional field follows...
\category{H.3.4}{Information Storage and Retrieval}{Systems and Software}[Current awareness systems]
%\terms{Access Control, Mobile privacy and security}
\keywords{Access Control, Semantic Web, Context-Aware Computing}