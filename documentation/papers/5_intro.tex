\section{Introduction}
\label{intro}
Mobile devices have become ubiquitous due to its low cost and Android is the biggest player in the market. Latest reports from Google boasts of more than a billion active 30 day user~\cite{Engadget_market_share}. According to the International Data Corporation's Worldwide Quarterly Mobile Phone Tracker report Android has a 85\% market share in the smartphone category. Apps from the Google Play Store and a variety of other app-stores like Amazon App Store and Samsung Galaxy Apps provide a plethora of ways through which Android users can get their apps~\cite{Online_App_Stores}. According to Statista~\cite{Android_app_number}, as of July 2015, there are more than 1.6 million android apps in the Google Play Store. 

The proliferation of smartphones has led to the popularity of the BYOD(Bring-Your-Own-Device) paradigm, whereby users' use their personal devices for corporate purposes. Naturally, this creates a greater need to ensure strong access control mechanisms for the data on such devices. In certain domains the access control needs are of a critical nature. For example in the category of Medical and Health \& fitness apps it is essential that user data and if being used in hospitals, corporate data security be maintained to the utmost level. Hospitals today use various hardware devices that are smart enough to communicate with smartphones and may even contain sensitive medical data. In addition to that android apps are capable of collecting a huge amounts of data about the smartphone user, often without the knowledge of the user. 

There have been multiple attempts at achieving the goal of properly managing access control on mobile (Android) devices. Efforts have been made by the open source community through the XPrivacy project (needs a rooted phone), the Privacy Guard project (available on Cyanogenmod, a custom Android ROM), the PDroid application (needs a rooted device). Research project by Conti et. al.~\cite{conti2011crepe}(CRePe), Enck et al.~\cite{enck2010taintdroid}(TaintDroid) and Jagtap et al.~\cite{Jagtap2011Privacy}(Preserving Privacy in Context-Aware Systems) have made similar efforts. CRePE described a system where security policy enforcement was carried out based on context of the smart phone. TaintDroid was a research effort where the data flow on an Android device was studied to figure out when sensitive data left the system via an untrusted application. The work of Jagtap et al.~\cite{Jagtap2011Privacy} focused on constraining data flow in a context-aware system using a policy-based framework. A related work by Ghosh et al.~\cite{ghosh2012privacy} used a similar policy driven approach to constrain application permissions based on context. A %Given the abundance of works in the access control area, its understandable that the latest version of Android (Marhsmallow), embraced a new permission model. Android now boasts of a runtime, on-demand permission acquisition model, similar to iOS, the other leading smartphone platform. Such a change, although welcome, still remains inadequate with respect to context based data flow control. Android has attempted to make the transition for its developers and users simple by grouping the control into logical groups but it still leaves a lot to desire.

In this paper we focus on a custom permissions created by app developers. These permissions are there to protect the app developers data on their own content providers. It is advised by Google that if an app developer creates a content provider for allowing access to it's own data, they should create a permission to control access to it. However, this requirement is not a stringent one and one might simply ignore creating such a permission.

%Some of them include Blood glucose monitoring system[5], Blood pressure monitor[6], Heart monitor[7], Fitness accessories like Fitbit, Breathalyzer[4] etc. All these accessories have an associated Android application which interacts with the physical device, process the data, display it and stores relevant information. Another category of applications are those used by hospitals and its employees to manage information about the patients, doctors etc. The important point is that all these sensitive information are being received, processed, stored and possibly updated to cloud storage by these applications. The fact that personal and sensitive data is handled by these devices presents a potential security risk. With more than 1.6 million apps to choose from, let alone a layman user, even a computer expert will have difficulty to understand what are benign and what are malicious. One of the most traditional ways to look at this problem is to use signature based malware detection, in which once a malware application is detected, signature will be created once and the signatures are updated to a common database. Any future appearance of the same malware would be detected using these signatures. The classical problem with this approach is the occurrence of mutating malwares which changes it’s signature every time it is spread and it inability to detect future similar attacks. Apart from this Android apps has issues on its own. One of the main problem is the high volatility of the app markets. Since unlike Apple store, Google play store is not verified and hence new applications becomes available and taken out quite frequently. Yet another issue is repackaging popular application with malware. Often these repackaged applications are uploaded to other local App-stores where the original application is not available. Hence we require a faster and dynamic setup to detect unsafe or rather potentially unsafe applications. [Paper to be referred Dr B]Even though not totally accurate, there are some intuitions which we can utilize from the meta-data available with each application. The different meta-data available with each of these applications include App description, Permissions required, Number of downloads, Broad Application categories etc. As an example of our intuition let us take the case of “brightestlight” application, which is available in Google Play Store. The description of the application says that it is a flash light app, which shows some unobtrusive Ads. But on analyzing the Permissions required, we can see that the application requests permissions for location (accurate GPS and approximate network based), file manipulation (read, modify and delete USB storage), camera (curiously it includes auto-focus permission) etc. This does seem suspicious. But still we cannot say that the application is unsafe, rather we need further investigation. 

\section{Related Work}
\label{RelatedWork}
\hl{
Playdrone : Crawls Playstore- how playstore evolved - source code analysis of library usage - similar app detection-secret authentication key storage (can be found by decompilation)
(1) native libraries are heavily used by popular Android applications, limiting the benefits of Java portability and the ability of Android server overloading systems to run these applications, (2) 25\% of Google Play is duplicative application content, and (3) Android applications contain thousands of leaked secret authentication keys which

Andradar : First, we can discover malicious applications in alternative markets, second, we can expose app distribution strategies used by malware developers, and third, we can monitor how different markets react to new malware. To identify and track malicious apps still available in a number of alternative app markets.

Android Security : discuss the Android security enforcement mechanisms, threats to the existing security enforcements and related issues, malware growth timeline between 2010 and 2014, and stealth techniques employed by the malware authors, in addition to the existing detection methods. This review gives an insight into the strengths and shortcomings of the known research methodologies and provides a platform, to the researchers and practitioners, toward proposing the next-generation Android security, analysis, and malware detection techniques.

ANDRUBIS:, a fully automated, publicly available and comprehensive analysis system for Android apps. ANDRUBIS combines static analysis with dynamic analysis on both Dalvik VM and system level, as well as several stimulation techniques to increase code coverage. 

changes in the malware threat landscape and trends amongst goodware developers. Dynamic code loading, previously used as an indicator for malicious behavior, is especially gaining popularity amongst goodware App analysis for astma!!

App behavoir against description CHABADA tool clustering apps by description topics, and identifying outliers
by API usage within each cluster, our CHABADA approach effectively
identifies applications whose behavior would be unexpected
given their description.
Recommendations for android eco system}