\section{Introduction}
\label{intro}
As the number of internet connected mobile users increases, and confidential information is increasingly stored on these devices, they become attractive targets of attack. McAfee Labs 2014 report~\footnote{McAfee Lbs: Threats report~\url{http://goo.gl/oS8922}} predicted that mobile technologies would see an escalation of attacks due to openly available mobile malware source code. Another reason is the incentive -- mobiles are being used as payment devices at point of Sale (e.g. Google Wallet, Apple Pay). 

On the corporate end of things, incentive is also provided by  Bring-Your-Own-Device (BYOD) policies that permit the same user owned device to be used both within and outside the corporate firewall, and allows applications to run on the device without restrictions. ZDNet\footnote{Research: 74 percent using or adopting BYOD~\url{http://goo.gl/jldRtt}} reported about a study conducted by Tech Pro in November 2014, which showed that 74\% of participating organizations allow their employees to bring their own devices in to company premises, and that  60\% of the organizations allowed employees to use personal devices to access company networks and data. Another 14\% planned to allow the same within a year.  The potential attack surface openend by such policies is also well recognized.  A study conducted by Dimensional Research ~\footnote{The impact of mobile devices on information security~\url{http://goo.gl/pl2HVU}} reported that  67\% of the respondents believed that securing corporate information was the greatest BYOD challenge. 

The BYOD scenario can be used to motivate a key challenge in access control for mobile devices. Depending on the context of the usage, the access rights can change. For instance, it might be permissible to send some data over the corporate VPN, but not have it uploaded to Facebook! It might be OK to use the camera generally, but not inside the company facility. Reporting GPS locations to the platform provider (e.g. Google, Apple) might be fine in general, but not when inside a Sensitive Compartmented Information Facility (SCIF). Note that the current ``permit at install'' model of most mobile OSs is inadequate to this task. It also has no way to indicate if the sought for permissions are relevant to the task of the requester, and most users cannot make this judgement themselves. An example of this was observed in 2013, when it was revealed that the Federal Trade Commission had found out that an app~\footnote{Brightest Flashlight Free~\url{http://goo.gl/qBSKrs}} was collecting user location data. The app had over fifty million downloads and nearly a million `5-star' ratings on the Google Android Play Store but was deceiving users by collecting their location and identity and sharing the same with third-party advertising networks. Solving the Access control on mobile devices is a therefore a key problem.

Naturally, there have been multiple attempts at achieving the goal of managing access control on mobile (Android) devices. Efforts have also been made by the open source community to create systems that allow users to control permissions on the Android platform. Some of these efforts require custom ROMs or rooted devices. Most prominent examples of these efforts include the XPrivacy project (needs a rooted phone), the Privacy Guard project (available on Cyanogenmod, a custom Android ROM), the PDroid application (needs a rooted device). In 2012 Google introduced AppOps as part of Android 4.3. This feature was the first customizable permission model seen on the Android platform. AppOps made it possible to control the data flow on Android in a way that had not been possible before. Unfortunately, this option was removed with the release of Android version 4.4.2 and is only accessible on a ``rooted'' mobile device. 

There have also been research efforts that aim to use either data tainting or policy driven approaches to address this problems. Some of these efforts include work done by Conti et. al.~\cite{conti2011crepe}(CRePE), Enck et al.~\cite{enck2010taintdroid}(TaintDroid) and Jagtap et al.~\cite{Jagtap2011Privacy}(Preserving Privacy in Context-Aware Systems). CRePE described a system where security policy enforcement was carried out based on context of the smart phone. TaintDroid was a research effort where the data flow on an Android device was studied to figure out when sensitive data left the system via an untrusted application. The work of Jagtap et al.~\cite{Jagtap2011Privacy} focused on constraining data flow in a context-aware system using a policy-based framework. A related work by Ghosh et al.~\cite{ghosh2012privacy} used a similar policy driven approach to constrain application permissions based on context.

All these studies indicate that there has been a fair amount of effort focused on access-control on mobile devices. However, most of these works left out the research on how to create the policies that are to be enforced on the mobile devices. More recently efforts~\cite{Benisch2011,Sadeh2009,lin2014soups,liu2014www} have been made to determine the permissions a user or a class of users might prefer. In this paper, we focus on automatically capturing a user's access control policies that are fine-grained, dynamic and context-dependent.  We also present an end-to-end system design that allows enforcement of such rules on an mobile device. In order to carry out these two tasks, we present \textbf{\textsc{Mithril}}\footnote{Mithril is a reference to a precious, lightweight and extremely strong silvery metal from the Lord of the Rings which protected its wearer, Frodo, from life threatening dangers:~\url{http://goo.gl/sh2jXG}}, a system for capturing fine-grained (context dependent) user policies and managing privacy and security on mobile devices using said policies. We adopt an iterative model for capturing specific user policy. We begin the process by using an initial default policy and observe all violations of said policy that happens on the mobile device. We use these violation data to improve the policy using user-feedback. 

The rest of the paper is organized as follows. We start with a system overview in the next section. The third section provides the details of our methodology for capturing specific user policy on a mobile device. We present our experimental evaluation methods in the fourth section followed by the results of our study and discussion of our results in the fifth section. We contrast our work with the related work found in the literature in the sixth section. Finally we conclude the paper with a summary of the current work and our future research goals.
%} %End of 'Large' font introduction section and abstract