\section{Experimental evaluation}
\label{eval}
As the main focus of this paper is to capture specific user policies, we focus our evaluation on the same. Following are the ways we have evaluated our system. We installed \textsc{Mitrhil} on several user devices and on each of those devices we started with an initial default policy. For the sake of simplicity we will denote all default policies with the letter P. For the purposes of our experiment we requested ten graduate students to provide us their feedback. Before we ask them to use our system, we also ask them to use a web app to modify the default graduate student policy to specify their own specific policy. 

As stated before, \textsc{Mitrhil} has two operating modes, observer and enforcer. In the observer mode the policy P is taken as a reference point and app activity on the mobile device is monitored. Any violation, as detected by the policy decision module is recorded and then at a pre-defined time period, the user is presented with these potential violations. The user then modifies the rules, if necessary. In our experiments we record on a per iteration basis, the following statistics:-
\begin{itemize}
	\item Number of rule violations recorded.
	\item Number of rule changes made by user.
	\item Number of condition changes made per rule.
	\item Distance from their ideal policy.
\end{itemize}

Through our evaluations we are trying to find out the fraction of the users who are able to reach their ideal policy as defined at the start of the experimental evaluation. 


% for data was The Mithril records the violations, of the default policy, happening on the mobile device. At the end of the observer operating mode, the user is queried about the violations and necessary rule modifications are carried out. At this stage \textsc{Mithril} switches its operating mode to enforcer. In the enforcer mode \textsc{Mithril} continues to capture violations of the applicable policy. This continued observation facilitates a periodic feedback mechanism.


  

%The policy enforcement module is the entry point for In Android 4.3 Google introduced a hidden feature in the Android framework called AppOps~\cite{appops22013}. AppOps allowed user's to control the permissions available to an app. AppOps also provides us with logs for data access requests by apps, a methodology discovered by Almuhimedi et. al~\cite{almuhimedi2014your}. , We use the AppOps logs to determine the number of times a This methodology was used by We use AppOps to enforce the policies that are applicable on the device at any time. Despite its advantages, AppOps is not automated. A user would have to deliberately go and change permissions for a specific app if they want to have data privacy and security on their devices. Therefore, our system provides the added advantage of fine-grained policy control which is dependent on the user's current context. In order to determine the applicable policy The policy enforcement asks for a policy decision from the Policy Decision Module.
 
%As mentioned before the goal of the end-to-end system is to find out what policy rules are required to be implemented on a user's phones. Additionally it also aims to implement these semantically-rich set of policy rules, which are defined using the user's context, to control the data flow on a smart-phone. 



%In this paper we focus on the first portion of the problem. We start from a general policy applicable to a user-category and over a certain number of iterations reach a state with a set of specific policy rules applicable to a particular user belonging to said user-category. Our system contains a pre-installed set of policy rules for the user categories. For the purposes of demonstration of our system we have created the following user groups: Working Professional, Academician, Adult Student, Under-Age Student, Retired person. We ran experiments to show that using an ontology for modeling context and OWL-DL inference mechanisms it is possible to go from a generic policy P, applicable to a user-category to a user's specific policy P'.

%During the installation of the system on a user's mobile we allow a device administrator to select the user-category(either a supervising adult or user themselves complete this selection). The administrator was also requested to specify how frequently should the user be requested to provide rule modification feedback. The process for obtaining these policies is beyond the scope of the present work. Multiple research projects by Norman Sadeh et. al.~\cite{Benisch2011,Sadeh2009,lin2014soups,liu2014www} have been conducted to capture user privacy policy. Therefore, we make an assumption that an initial policy with reasonably high accuracy can be obtained. \hlc{These works refer to policies as permissions that user's would and would not allow. As discussed in earlier sections (discuss the need for dynamic context driven fine grained policy), we are emphasizing the need for fine-grained context driven policies.}



\subsection{Data collection}
\label{datacol}
%As part of Android 4.3 Google created an option under settings which was the first time that we observed a departure from Android's standard ``take-it-or-leave-it''. This option was called AppOps. However, with Android version 4.4.2 this option was hidden from the settings menu. It was still possible to access this menu on a ``rooted'' mobile device. However, a rooted device created other security issues (its like using a Linux operating system with root privileges) and is therefore not an ideal solution. We used mobile devices with Android 4.3 to allow us to access AppOps logs. These logs were used to detect activities that were happening on the phone.

%We ran our system in two different modes. The first one being ``rule capture'' mode and the second was the ``rule implementation'' mode. In the capture mode our system simply collected all the ``violations'' of the initial policy that was set up on the device. Given an initial policy for a user-category, we started our data collection and observed the ``violations'' of the initial policy. The violations were then presented to the user as per the user's specified feedback frequency. After the first iteration, the number of changes made were recorded. This count would be used to measure the utility and viability of the system. At this point the system mode shifts to implementation mode. The implementation of the policy rules on Android devices has been well-explored~\cite{conti2011crepe} and is beyond the scope of this paper. Even in the implementation mode the violations, if any, of the new policy are still observed. This iterative model is  followed till the number of changes made in each iteration falls below a threshold value set by the device administrator. 

%For each unique violation on the phone the rule is presented to the user in the format seen in Figure~\ref{fig:architecturePolicy}. The additional information provided to the user include a risk factor assessment. The risk factor assessment is computed based on the number of permissions requested by the app, the types of permissions requested, the apps Play Store Rating, Download Count and Maturity Rating. This helps the user make a more informed decision.

\section{Results}
\label{results}